\documentclass{ETHExercise}
\usepackage{hyperref}
\usepackage{enumitem}
\usepackage{graphicx}
\usepackage{listings}
\usepackage{xcolor}
\usepackage{float}
\usepackage{subfig}
\usepackage{adjustbox}
\usepackage[margin=2cm]{geometry}
\usepackage{fancyhdr}
\usepackage{capt-of}
\usepackage{tabularx}
\usepackage{bm}
\usepackage{amsmath}
\usepackage{datetime}
\usepackage{listings}
\usepackage{subcaption}
\usepackage[ruled,vlined]{algorithm2e}




\pagestyle{fancy}
\fancyhf{}
\fancyfoot[L]{263-5902-00L W23 / Lab report}
\fancyfoot[C]{\thepage}
\fancyfoot[R]{Computer Science\\ETH Zürich}
\renewcommand{\footrulewidth}{0.05pt}
\renewcommand{\headrulewidth}{0pt}
\renewcommand{\footrule}{\hrule width\textwidth height\footrulewidth \vskip 2pt}
\newcommand{\timestamp}{\ddmmyyyydate\today \,\,- \currenttime h}

\usepackage{mdframed}
\usepackage{etoolbox}

\newtoggle{showA}
\toggletrue{showA}





\hypersetup{colorlinks,urlcolor=blue}
\lstset{
  backgroundcolor=\color{lightgray!20},
  basicstyle=\small\ttfamily,
  keywordstyle=\color{blue}\ttfamily,
  language=C,
  numbers=left,
  stepnumber=1,
  tabsize=4,
  showspaces=false,
  showstringspaces=false,
}
\title{Lab 5}
\author{Filippo Ficarra}
\begin{document}


\lectureheader{Prof. M. Pollefeyes}
{}
{\Large Computer Vision}{Fall 2023}
\begin{center}
    {\Huge Filippo Ficarra: Lab 6 - Condensation Tracker}\\
      \quad\newline
      fficarra@student.ethz.ch, 22--938--062.\\
      \quad\newline
      \timestamp
      \end{center}



\section{Implementation}
In this exercise we implemented the CONDENSATION algorithm, a sampled-based solution of the recursive Bayesian filter.
We needed to implemt 5 functions:
\begin{itemize}
  \item color\_histogram
  \item propagate
  \item observe
  \item estimate
  \item resample
\end{itemize}
All of this functions are implemented in the corresponted .py files.
\subsection{Color histogram}
\begin{lstlisting}[language=Python, caption=color histogram]
  def color_histogram(xmin, ymin, xmax, ymax, frame, hist_bin):
    bounding_box = frame[ymin:ymax, xmin:xmax]
    hist_channels = []
    
    for channel in range(bounding_box.shape[2]):
        hist, _ = np.histogram(bounding_box[:,:,channel], bins=hist_bin)
        hist_channels.append(hist)
        
    histogram = np.concatenate(hist_channels)
    norm_histogram = histogram / np.sum(histogram)

    return norm_histogram

\end{lstlisting}

In this function I am creating for each channel (RGB)
an histogram with hist\_bin number of bins for the pixels. What the 
function does is to take the patch of the image described by the bounding box,
create the histogram for each channel, concatenate and then normalize.

\newpage

\subsection{Propagate}

\begin{lstlisting}[language=Python, caption=propagate]
  
  def propagate(particles, frame_height, frame_width, params):
    delta_t = 1
    sigma_p = params["sigma_position"]
    sigma_v = params["sigma_velocity"]

    A = np.identity(particles.shape[1])
    model_noise_std = np.array([sigma_p, sigma_p])

    if params["model"] == 1:
        A[0, 2] = delta_t
        A[1, 3] = delta_t
        model_noise_std = np.array([sigma_p, sigma_p, sigma_v, sigma_v])

    w = np.random.randn(particles.shape[0], particles.shape[1]) * model_noise_std


    particles = np.matmul(particles, A.T) + w


    particles[:, 0] = np.clip(particles[:, 0], 0, frame_width - 1)
    particles[:, 1] = np.clip(particles[:, 1], 0, frame_height - 1)
      
  return particles

\end{lstlisting}
This is the pivotal function of the algorithm. Here each sample $s'_{t-1}$ 
is propagated with the following formula: $s^{(n)}_t = As'^{(n)}_{t-1} + w^{(n)}_{t-1}$.
First, each sample is of this form: $s=\{x, y, \dot{x}, \dot{y}\}$. The matrix 
Thge matrix A is taken as the identity for the no motion case, or the identity 
plus an identity with the diagonal shifted up and multiplied by dt.
For the no motion model the sampling gives this result:
\begin{center}
  $A = \begin{bmatrix}
    1 & 0\\
    0 & 1
  \end{bmatrix} \implies (x^{(n)}_t, y^{(n)}_t)^\top = \begin{bmatrix}
    1 & 0\\
    0 & 1
  \end{bmatrix} (x^{(n)}_{t-1}, y^{(n)}_{t-1})^\top + ({\sigma_p}^{(n)}_{t-1}, {\sigma_p}^{(n)}_{t-1})^\top=$\\
  \[
  \begin{cases}
    x^{(n)}_t = x^{(n)}_{t-1} + {\sigma_p}^{(n)}_{t-1}&\\
    y^{(n)}_t = y^{(n)}_{t-1} + {\sigma_p}^{(n)}_{t-1}&
  \end{cases}
  \]
\end{center}

For the constant velocity we are using the following contruction:
\begin{center}
  $A = \begin{bmatrix}
    1 & 0 & dt & 0\\
    0 & 1 & 0 & dt\\
    0 & 0 & 1 & 0\\
    0 & 0 & 0 & 1
  \end{bmatrix} \implies (x^{(n)}_t, y^{(n)}_t, \dot{x}^{(n)}_t, \dot{y}^{(n)}_t)^\top = \begin{bmatrix}
    1 & 0 & dt & 0\\
    0 & 1 & 0 & dt\\
    0 & 0 & 1 & 0\\
    0 & 0 & 0 & 1
  \end{bmatrix} (x^{(n)}_{t-1}, y^{(n)}_{t-1}, \dot{x}^{(n)}_{t-1}, \dot{y}^{(n)}_{t-1})^\top + ({\sigma_p}^{(n)}_{t-1}, {\sigma_p}^{(n)}_{t-1}, {\sigma_v}^{(n)}_{t-1}, {\sigma_v}^{(n)}_{t-1})^\top=$\\
  \[
  \begin{cases}
    x^{(n)}_t = x^{(n)}_{t-1} + \dot{x}^{(n)}_{t-1} dt + {\sigma_p}^{(n)}_{t-1}& \\
    y^{(n)}_t = y^{(n)}_{t-1} + \dot{y}^{(n)}_{t-1} dt + {\sigma_p}^{(n)}_{t-1}& \\
    \dot{x}^{(n)}_{t} = \dot{x}'^{(n)}_{t-1} + {\sigma_v}^{(n)}_{t-1} &\\
    \dot{y}^{(n)}_{t} = \dot{y}'^{(n)}_{t-1} + {\sigma_v}^{(n)}_{t-1}
  \end{cases}
  \]
\end{center}
I assumed dt to be 1, and w is a random matrices with covariance matrix given by $\sigma_p$ and $\sigma_v$.
At the end the particles are clipped to be in the frame.

\subsection{Observe}
\begin{lstlisting}[language=Python, caption=observe]
  def observe(particles, frame, bbox_height, bbox_width, hist_bin, hist, sigma_observe):
    particles_w = np.zeros(particles.shape[0])
    for i, particle in enumerate(particles):
        x_center = particle[0]
        y_center = particle[1]
        
        x_min = np.clip(int(x_center - bbox_width/2), 0, frame.shape[1]-1)
        y_min = np.clip(int(y_center - bbox_height/2), 0, frame.shape[0]-1)
        x_max = np.clip(int(x_center + bbox_width/2), 0, frame.shape[1]-1)
        y_max = np.clip(int(y_center + bbox_height/2), 0, frame.shape[0]-1)
        
        histogram = color_histogram(x_min, y_min, x_max, y_max, frame, hist_bin)
        
        chi_2 = chi2_cost(histogram, hist)
        
        particles_w[i] = 1/(np.sqrt(2*np.pi)*sigma_observe) * np.exp(-chi_2**2/(2*sigma_observe**2))
        
    particles_w = particles_w / np.sum(particles_w)
    
    return particles_w
\end{lstlisting}

For each particle we compute the Chi2 distance between the color histogram of the box around the particle
and the otriginal color histogram. Then each weight is computed using the gaussian function on this Chi2 distances.
Everything at the end is normalized to mantain this weight as probabilities.

\subsection{Estimate}

\begin{lstlisting}[language=Python, caption=estimate]
  def estimate(particles, particles_w):
    estimate = np.zeros(particles.shape[1])
    
    for i, particle in enumerate(particles):
        estimate += particle * particles_w[i]
        
    estimate = estimate / np.sum(particles_w)

    return estimate
\end{lstlisting}

This function compute the estimated particle, computing its weighted mean using the particle weights.

\subsection{Resample}
\begin{lstlisting}[language=Python, caption=resample]

  def resample(particles, particles_w):
      index = np.random.choice(particles.shape[0], particles.shape[0], p=particles_w)
      return particles[index], particles_w[index]
\end{lstlisting}
With this function then we resample the particles given the weights computed in the observe function.
\section{Results}

\subsection{Video 1}

The algorithm works pretty well with the first video since the movement 
is very simple and the color are pretty distinguishable. The tracker worked, in a sense, with the
default parameters but we, as we can see in the picture below, it tended to follow the wrist and it 
tended to loose the tracking almost at the end of the video, when the hand moved quicker.
Below the results for the video 1 and the default parameters.


\begin{figure}[!h]
  \minipage{0.3\textwidth}
    \includegraphics[width=\linewidth]{images/video_1_default/12.png}
  \endminipage
  \minipage{0.3\textwidth}
    \includegraphics[width=\linewidth]{images/video_1_default/19.png}
  \endminipage
  \minipage{0.3\textwidth}
    \includegraphics[width=\linewidth]{images/video_1_default/27.png}
  \endminipage\space
  \minipage{0.3\textwidth}
    \includegraphics[width=\linewidth]{images/video_1_default/34.png}
  \endminipage
  \caption{Video 1 with default parameters, no motion model}
\end{figure}

If we increase more the numbers of particles we should get more 
accurate results. As shown below the tracker gets better in the last part
of the video when it doesn't loose the hand. One of the problem is 
that the tracker still follows the wrist, and this can be due to the no motion
model or the hist\_bin, since the image contains the shadow of the hand. 

\begin{figure}[!h]
  \minipage{0.3\textwidth}
    \includegraphics[width=\linewidth]{images/video_1_default_300_particles/12.png}
  \endminipage
  \minipage{0.3\textwidth}
    \includegraphics[width=\linewidth]{images/video_1_default_300_particles/19.png}
  \endminipage
  \minipage{0.3\textwidth}
    \includegraphics[width=\linewidth]{images/video_1_default_300_particles/27.png}
  \endminipage\space
  \minipage{0.3\textwidth}
    \includegraphics[width=\linewidth]{images/video_1_default_300_particles/34.png}
  \endminipage
  \caption{Video 1 with default parameters, no motion model, 300 particles}
\end{figure}

\newpage
What happens if we play with other parameters like the number
of bins? Below I show that reducing the number of hist\_bins impact
on the tracking and make it fail. This is due to the fact that we are 
grouping very different colors together, and this makes the algorithm 
not robust to light changing for example.

\begin{figure}[!h]
  \minipage{0.3\textwidth}
    \includegraphics[width=\linewidth]{images/video_1_default_hist_bin8/12.png}
  \endminipage
  \minipage{0.3\textwidth}
    \includegraphics[width=\linewidth]{images/video_1_default_hist_bin8/19.png}
  \endminipage
  \minipage{0.3\textwidth}
    \includegraphics[width=\linewidth]{images/video_1_default_hist_bin8/27.png}
  \endminipage\space
  \minipage{0.3\textwidth}
    \includegraphics[width=\linewidth]{images/video_1_default_hist_bin8/34.png}
  \endminipage
  \caption{Video 1 with default parameters, no motion model, hist\_bin 8}
\end{figure}

\begin{figure}[!h]
  \minipage{0.3\textwidth}
    \includegraphics[width=\linewidth]{images/video_1_default_hist_bin4/12.png}
  \endminipage
  \minipage{0.3\textwidth}
    \includegraphics[width=\linewidth]{images/video_1_default_hist_bin4/19.png}
  \endminipage
  \minipage{0.3\textwidth}
    \includegraphics[width=\linewidth]{images/video_1_default_hist_bin4/27.png}
  \endminipage\space
  \minipage{0.3\textwidth}
    \includegraphics[width=\linewidth]{images/video_1_default_hist_bin4/34.png}
  \endminipage
  \caption{Video 1 with default parameters, no motion model, hist\_bin 4}
\end{figure}

\newpage
If we increase the number of hist\_bins to 32 we see
an improvement on the tracking. The model is able to track better 
the hand and not only the wrist. Also the model is 
more stable to the sudden disappearance of the hand at the end.
\begin{figure}[!h]
  \minipage{0.3\textwidth}
    \includegraphics[width=\linewidth]{images/video_1_default_hist_bin32/12.png}
  \endminipage
  \minipage{0.3\textwidth}
    \includegraphics[width=\linewidth]{images/video_1_default_hist_bin32/19.png}
  \endminipage
  \minipage{0.3\textwidth}
    \includegraphics[width=\linewidth]{images/video_1_default_hist_bin32/27.png}
  \endminipage\space
  \minipage{0.3\textwidth}
    \includegraphics[width=\linewidth]{images/video_1_default_hist_bin32/34.png}
  \endminipage
  \minipage{0.3\textwidth}
    \includegraphics[width=\linewidth]{images/video_1_default_hist_bin32/35.png}
  \endminipage
  \caption{Video 1 with default parameters, no motion model, hist\_bin 32}
\end{figure}

\newpage
While no motion model is working, thanks also to the sigma\_position,
the result can be improved using a constant velocity model. 
The only parameter changed here are the initial velocity to (-1, -5) 
,so the tracker is initialy directed to top left, and the sigma\_position
reduced to 10.


\begin{figure}[!h]
  \minipage{0.3\textwidth}
    \includegraphics[width=\linewidth]{images/video_1_constant_velocity_sigma_position_10_initial_velocity_-1_-5/12.png}
  \endminipage
  \minipage{0.3\textwidth}
    \includegraphics[width=\linewidth]{images/video_1_constant_velocity_sigma_position_10_initial_velocity_-1_-5/19.png}
  \endminipage
  \minipage{0.3\textwidth}
    \includegraphics[width=\linewidth]{images/video_1_constant_velocity_sigma_position_10_initial_velocity_-1_-5/27.png}
  \endminipage\space
  \minipage{0.3\textwidth}
    \includegraphics[width=\linewidth]{images/video_1_constant_velocity_sigma_position_10_initial_velocity_-1_-5/34.png}
  \endminipage
  \minipage{0.3\textwidth}
    \includegraphics[width=\linewidth]{images/video_1_constant_velocity_sigma_position_10_initial_velocity_-1_-5/35.png}
  \endminipage
  \caption{Video 1 with default parameters,constant velocity, initial velocity (-1, -5), sigma\_position 10}
\end{figure}


\subsection{Video 2}
The second video was more complex than the first one, since there is 
an object that performs occlusion and the background is not constant 
as in the first video. As first model I tried the one with 
default parameters. Below the results.

\begin{figure}[!h]
  \minipage{0.3\textwidth}
    \includegraphics[width=\linewidth]{images/video2_default/7.png}
  \endminipage
  \minipage{0.3\textwidth}
    \includegraphics[width=\linewidth]{images/video2_default/12.png}
  \endminipage
  \minipage{0.3\textwidth}
    \includegraphics[width=\linewidth]{images/video2_default/27.png}
  \endminipage\space
  \minipage{0.3\textwidth}
    \includegraphics[width=\linewidth]{images/video2_default/34.png}
  \endminipage
  \caption{Video 2 with default parameters, no motion model}
\end{figure}

The model almost track the hand for the whole video. As shown in the images,
we have high inaccurancy near the obstacle, but the model is still able to 
follow the hand.

Lets now try the constant velocity. Since the object is moving right without 
any y movement, we can set the inital velocity to (3, 0).

\begin{figure}[!h]
  \minipage{0.3\textwidth}
    \includegraphics[width=\linewidth]{images/video2_default_constant_velocity_inital_velovity_3_0/7.png}
  \endminipage
  \minipage{0.3\textwidth}
    \includegraphics[width=\linewidth]{images/video2_default_constant_velocity_inital_velovity_3_0/12.png}
  \endminipage
  \minipage{0.3\textwidth}
    \includegraphics[width=\linewidth]{images/video2_default_constant_velocity_inital_velovity_3_0/27.png}
  \endminipage\space
  \minipage{0.3\textwidth}
    \includegraphics[width=\linewidth]{images/video2_default_constant_velocity_inital_velovity_3_0/34.png}
  \endminipage
  \caption{Video 2 with default parameters, constant velocity model, initial velocity (3, 0)}
\end{figure}

This tracking fails, due to the occlusion. We can imporve the tracking sertting $\alpha$ to 0.5, 
so that the color histogram is updated with with the current histogram. Despite this change
the result are not so different from the constan motion.

\begin{figure}[!h]
  \minipage{0.3\textwidth}
    \includegraphics[width=\linewidth]{images/video_2_constant_velocity_alpha_05_initial_velocity_3_0/7.png}
  \endminipage
  \minipage{0.3\textwidth}
    \includegraphics[width=\linewidth]{images/video_2_constant_velocity_alpha_05_initial_velocity_3_0/12.png}
  \endminipage
  \minipage{0.3\textwidth}
    \includegraphics[width=\linewidth]{images/video_2_constant_velocity_alpha_05_initial_velocity_3_0/27.png}
  \endminipage\space
  \minipage{0.3\textwidth}
    \includegraphics[width=\linewidth]{images/video_2_constant_velocity_alpha_05_initial_velocity_3_0/34.png}
  \endminipage
  \caption{Video 2 with default parameters, constant velocity model, initial velocity (3, 0)}
\end{figure}

If we further change the parameters we can find an optimal setting for tracking. What performed best was 300 particles,
sigma\_position = 1, sigma\_velocity = 0.1, $\alpha = 0.5$ and initial velocity (4,0).
Below also a comparison of the result when using different sigma\_observe values.
\newpage
\begin{figure}[!h]
  \minipage{0.5\textwidth}
    \includegraphics[width=\linewidth]{images/video2_best_sigma_observe_01/34.png}
  \endminipage
  \minipage{0.5\textwidth}
    \includegraphics[width=\linewidth]{images/video2_best_sigma_observe_1/34.png}
  \endminipage
  \caption{Best params for Video 2, sigma\_observe 0.1 vs sigma\_observe 1}
\end{figure}

The second image seem to be a better tracking, but the problem is that, due to the high observation uncertainty,
the tracker tend to be ahead the hand and not tracking it for the middle part of the video.

If we increase sigma\_position to 30, the tracking result in failure as shown in the image.
\begin{figure}[!h]
  \centering
    \includegraphics[width=0.5\linewidth]{images/video2_best_sigma_position_30/28.png}
  \caption{Best params for Video 2, sigma\_position 30}
\end{figure}

The outcome of this experiment for now tell us that:
\begin{itemize}
  \item Reduced system noise in the model typically leads to more confidence in the predicted motion. 
  \item Higher system noise makes the predictions less reliable and more susceptible to errors.
  \item Lower measurement noise implies more accurate and reliable measurements of the object's position.
  This leads to better updates in the algorithm.
  \item Higher measurement noise introduces more uncertainty into the measurements.
  This makes the update less accurate.
  \item Constant velocity is very useful when we know that the object is moving at a constant
  velocity, since the assumption of constant velocity helps to estimate next frames in which the object
  is more likely to appear
\end{itemize}

\newpage

\subsection{Video3}

I started looking at the effect of video 3 with the best parameters of the video 2.
These are the results:
\begin{figure}[!h]
  \minipage{0.3\textwidth}
    \includegraphics[width=\linewidth]{images/video_3_best_param_2/2.png}
  \endminipage
  \minipage{0.3\textwidth}
    \includegraphics[width=\linewidth]{images/video_3_best_param_2/8.png}
  \endminipage
  \minipage{0.3\textwidth}
    \includegraphics[width=\linewidth]{images/video_3_best_param_2/18.png}
  \endminipage\space
  \minipage{0.3\textwidth}
    \includegraphics[width=\linewidth]{images/video_3_best_param_2/27.png}
  \endminipage
  \caption{Video 3 with best parameters for model 2}
\end{figure}
 Clearly the approach have some problems. First we notice that the initial velocity is not adequate for the 
 video in analysis. We can impove this just increasing the velocity on the x axis to 10.
 We see a slightly improvement on the tracking, but still the algorithm is not capable to follow the ball for the entire video.

\begin{figure}[!h]
  \minipage{0.3\textwidth}
    \includegraphics[width=\linewidth]{images/video_3_best_param_2_increased_velocity/2.png}
  \endminipage
  \minipage{0.3\textwidth}
    \includegraphics[width=\linewidth]{images/video_3_best_param_2_increased_velocity/8.png}
  \endminipage
  \minipage{0.3\textwidth}
    \includegraphics[width=\linewidth]{images/video_3_best_param_2_increased_velocity/18.png}
  \endminipage\space
  \minipage{0.3\textwidth}
    \includegraphics[width=\linewidth]{images/video_3_best_param_2_increased_velocity/27.png}
  \endminipage
  \caption{Video 3 with best parameters for model 2, initial velocity (10, 0)}
\end{figure}

This is because the ball changes direction suddenly, so the constant velocity is not able to follow, we can go back to the no motion model,
but we need to increase the system noise.

\begin{figure}[!h]
  \minipage{0.5\textwidth}
    \includegraphics[width=\linewidth]{images/video3_no_motion_position_1/25.png}
  \endminipage
  \minipage{0.5\textwidth}
    \includegraphics[width=\linewidth]{images/video_3_no_motion_position_10/34.png}
  \endminipage
  \caption{Video 3, no motion model, sigma\_position 1 vs 10, sigma\_observe 0.1}
\end{figure}

As expected the system noise, when increased, helps tracking in a no motion model. In the first 
image, due to the very low sigma\_position the tracker didn't move, while increasing its value 
gave an almost perfect tracking.
\newpage
We can also look at the effect of decrasing and increasing the observation uncertainty.

\begin{figure}[!h]
  \minipage{0.5\textwidth}
    \includegraphics[width=\linewidth]{images/video3_best_sigma_observe_005/34.png}
  \endminipage
  \minipage{0.5\textwidth}
    \includegraphics[width=\linewidth]{images/video3_best_sigma_observe_1/34.png}
  \endminipage
  \caption{Video 3, no motion model, sigma\_obeserve 0.05 vs 1.0}
\end{figure}

Summary:
\begin{itemize}
  \item Number of particles when increased makes the tracker more accurate
  since we have more samples from which apporximate the prediciton distribution
  \item Increasing the number of bins in a histogram color model provides a more detailed 
  representation of color information. This for better discrimination 
  between different colors and shades, potentially leading to more accurate 
  object recognition and tracking.
  \item Allowing appearance model updating enables the tracker to adapt to 
  changes in the object's appearance over time. This adaptation helps in 
  handling variations in lighting conditions, occlusions etc...
\end{itemize}
\end{document}